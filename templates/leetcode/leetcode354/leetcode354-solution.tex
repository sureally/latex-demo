% !TEX program = pdflatex
\documentclass[a4paper]{article}

% 颜色
\usepackage{xcolor}

% 代码插入
\usepackage{listings}
% 设置代码格式
\lstset{
    basicstyle=\small\menlo,          % 基本代码风格
    commentstyle=\color{gray} \itshape\menlo,     % 设置注释颜色
    keywordstyle=\color{blue} \small\menlo \bfseries,       % 设置keyword颜色
    stringstyle=\small\menlo,  % 字符串风格
    breaklines,                     % 自动换行
    flexiblecolumns,                % 别问为什么,加上这个
    numbers             =   left,   % 行号的位置在左边
    showspaces          =   false,  % 是否显示空格,显示了有点乱,所以不现实了
    numberstyle=\tiny\menlo,
    showstringspaces    =   false,
    rulesepcolor= \color{gray},     %代码块边框颜色
    captionpos          =   t,      % 这段代码的名字所呈现的位置,t指的是top上面
    frame               =   lrtb,   % 显示边框
    escapeinside=``, % 设置逃逸字符,可以冲lstlisting环境跳回到LaTex环境
    xleftmargin=2em,
    xrightmargin=2em, 
    aboveskip=1em
}

% 指定字体
\usepackage{fontspec}
\newfontfamily{\menlo}{Menlo}

% 字体包
\usepackage[UTF8]{ctex}
% 默认字体
\setmainfont[]{Times New Roman}
\setsansfont[]{Arial}
\setmonofont{Monaco}        % 等宽字体,一般用于排版代码

% 设置每页上下左右页间距
\usepackage{geometry}
\geometry{left=2.5cm,right=2.5cm,top=2.5cm,bottom=2.5cm}


% 使用行间距
\usepackage{setspace}
\renewcommand{\baselinestretch}{1.5}

% 超链接
\usepackage{hyperref}
\hypersetup{
    colorlinks=true,
    linkcolor=blue,
    filecolor=blue,      
    urlcolor=blue,
    citecolor=cyan,
}

\title{LeetCode354-俄罗斯套娃信封问题}

\begin{document}
\maketitle

\section{题目描述}
给定一些标记了宽度和高度的信封,宽度和高度以整数对形式$(w,h)$出现。当另一个信封的宽度和高度都比
这个信封大的时候,这个信封就可以放进另一信封里,如同俄罗斯套娃一样。

请计算最多能有多少个信封能组成一组“俄罗斯套娃”信封(即可以把一个信封放到另一个信封里面)。

\noindent
\textbf{说明:}

不允许选择信封。

\noindent
\textbf{示例:}

\begin{lstlisting}[language=bash]
    `\textbf{Input:}` envelopes=[[5,4],[6,4],[6,7],[2,3]]
    `\textbf{Output:}` 3
    `\textbf{Explanation:}` The maximum number of envelopes you can Russian doll is 3 ([2,3] => [5,4] => [6,7]).
\end{lstlisting}



\section{解法}

\subsection{方法1: 排序+最长递增子序列}

该问题为最长递增子序列的二维问题。

我们要找的最长子序列,且满足 $seq[i+1]$ 中的元素大于 $seq[i]$ 中的元素。

该问题是输入按任意顺序排列的--我们不能直接套用标准的LIS算法,需要先对数据进行排序。我们如何对
数据进行排序,以便我们的LIS算法总能找到最佳答案?

我们可以在\href{https://leetcode-cn.com/problems/longest-increasing-subsequence/
    ?utm_source=LCUS&utm_medium=ip_redirect_q_uns&utm_campaign=transfer2china}{这里}
找到最长递增子序列的解决办法。

\noindent
\textbf{算法:}

假设我们知道了信封套娃顺序,那么从里向外的顺序必须是按 $w$ 升序排序的子序列。

在对信封按 $w$ 进行排序以后,我们可以找到 $h$ 上最长递增子序列的长度。

我们考虑输入 $[[1,3],[1,4],[1,5],[2,3]]$,如果我们直接对 $h$ 进行 LIS 算法,我们将会
得到 [3,4,5],显然这不是我们想要的答案,因为 $w$ 相同的信封是不能够套娃的。

为了解决这个问题。我们可以按 $w$ 进行升序排序,若 $w$ 相同则按 $h$ 降序排序。则上述输入排序后为
$[[1,5],[1,4],[1,3],[2,3]]$,再对 $h$ 进行 LIS 算法可以得到 $[5]$,长度为 $1$,是正确的
答案。这个例子可能不明显。

我们将输入改为 $[[1,5],[1,4],[1,2],[2,3]]$。则提取 $h$ 为 $[5,4,2,3]$。我们对 $h$ 进行
LIS 算法将得到 $[2,3]$,是正确的答案。

\begin{lstlisting}[language=java]
class Solution {

    public int lengthOfLIS(int[] nums) {
        int[] dp = new int[nums.length];
        int len = 0;
        for (int num : nums) {
            int i = Arrays.binarySearch(dp, 0, len, num);
            if (i < 0) {
                i = -(i + 1);
            }
            dp[i] = num;
            if (i == len) {
                len++;
            }
        }
        return len;
    }

    public int maxEnvelopes(int[][] envelopes) {
        // sort on increasing in first dimension and decreasing in second
        Arrays.sort(envelopes, new Comparator<int[]>() {
            public int compare(int[] arr1, int[] arr2) {
                if (arr1[0] == arr2[0]) {
                    return arr2[1] - arr1[1];
                } else {
                    return arr1[0] - arr2[0];
                }
           }
        });
        // extract the second dimension and run LIS
        int[] secondDim = new int[envelopes.length];
        for (int i = 0; i < envelopes.length; ++i) secondDim[i] = envelopes[i][1];
        return lengthOfLIS(secondDim);
    }
}
\end{lstlisting}

\noindent
\textbf{复杂度分析: }

\begin{itemize}
    \item 时间复杂度: $O(N \log N)$。其中,$N$ 是输入数组的长度。
    排序和LIS算法都是 $O(N \log N)$。
    \item 空间复杂度:$lis$ 函数需要一个数组 $dp$,它的大小可达 $N$,另外,我们使用的排序
    算法也需要额外的空间。
\end{itemize}

\end{document}
