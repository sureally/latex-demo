% !TEX program = pdflatex
\documentclass[a4paper]{article}

% 代码插入
\usepackage{listings}
% 设置代码格式
\lstset{
    basicstyle          =   \sffamily,          % 基本代码风格
    keywordstyle        =   \bfseries,          % 关键字风格
    commentstyle        =   \rmfamily\itshape,  % 注释的风格,斜体
    stringstyle         =   \ttfamily,  % 字符串风格
    breaklines,                     % 自动换行
    flexiblecolumns,                % 别问为什么,加上这个
    numbers             =   left,   % 行号的位置在左边
    showspaces          =   false,  % 是否显示空格,显示了有点乱,所以不现实了
    numberstyle         =   \zihao{-5}\ttfamily,    % 行号的样式,小五号,tt等宽字体
    showstringspaces    =   false,
    captionpos          =   t,      % 这段代码的名字所呈现的位置,t指的是top上面
    frame               =   lrtb,   % 显示边框
    escapeinside=``, % 设置逃逸字符,可以冲lstlisting环境跳回到LaTex环境
    xleftmargin=2em,
    xrightmargin=2em, 
    aboveskip=1em
}

% 指定字体
\usepackage{fontspec}
\newfontfamily{\menlo}{Menlo}


% 字体包
\usepackage[UTF8]{ctex}
% 默认字体
\setmainfont[]{Times New Roman}
\setsansfont[]{Arial}
\setmonofont{Monaco}        % 等宽字体,一般用于排版代码

% 设置每页上下左右页间距
\usepackage{geometry}
\geometry{left=2.5cm,right=2.5cm,top=2.5cm,bottom=2.5cm}


%使用行间距
\usepackage{setspace}
\begin{spacing}{1.3}
\end{spacing}

\title{LeetCode354-俄罗斯套娃信封问题}

\begin{document}
\maketitle

\section{题目描述}
给定一些标记了宽度和高度的信封,宽度和高度以整数对形式$(w,h)$出现。当另一个信封的宽度和高度都比
这个信封大的时候,这个信封就可以放进另一信封里,如同俄罗斯套娃一样。

请计算最多能有多少个信封能组成一组“俄罗斯套娃”信封(即可以把一个信封放到另一个信封里面)。

\noindent
\textbf{说明:}

不允许选择信封。

\noindent
\textbf{示例:}

\begin{lstlisting}[language=bash,numbers=left,numberstyle=\tiny\menlo,basicstyle=\small\menlo]
    `\textbf{Input:}` envelopes=[[5,4],[6,4],[6,7],[2,3]]
    `\textbf{Output:}` 3
    `\textbf{Explanation:}` The maximum number of envelopes you can Russian doll is 3 ([2,3] => [5,4] => [6,7]).
\end{lstlisting}



\section{解法}


\end{document}
